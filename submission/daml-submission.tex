\documentclass[acmsmall]{acmart}
\settopmatter{printacmref=false} % Removes citation information below abstract
\renewcommand\footnotetextcopyrightpermission[1]{} % removes footnote with conference information in first column
\pagestyle{plain} % removes running headers
\usepackage[parfill]{parskip}

% defining the \BibTeX command - from Oren Patashnik's original BibTeX documentation.
\def\BibTeX{{\rm B\kern-.05em{\sc i\kern-.025em b}\kern-.08emT\kern-.1667em\lower.7ex\hbox{E}\kern-.125emX}}

% These commands are for a PROCEEDINGS abstract or paper.
\copyrightyear{2019}
\acmYear{2019}
\setcopyright{none}

% This is for conference proceedings published in book form.
\acmConference[Compose]{Compose Conference}{June 24--25, 2019}{NYC, NY}

% These commands are for a JOURNAL article.
%\setcopyright{acmcopyright}
%\acmJournal{TOG}
%\acmYear{2018}\acmVolume{37}\acmNumber{4}\acmArticle{111}\acmMonth{8}
%\acmDOI{10.1145/1122445.1122456}

%
% Submission ID.
% Use this when submitting an article to a sponsored event. You'll receive a unique submission ID from the organizers
% of the event, and this ID should be used as the parameter to this command.
%\acmSubmissionID{123-A56-BU3}

%
% The majority of ACM publications use numbered citations and references. If you are preparing content for an event
% sponsored by ACM SIGGRAPH, you must use the "author year" style of citations and references. Uncommenting
% the next command will enable that style.
%\citestyle{acmauthoryear}

\begin{document}

\title{DAML : A Haskell based language for Digital Contracts}

%
% The "author" command and its associated commands are used to define the authors and their affiliations.
% Of note is the shared affiliation of the first two authors, and the "authornote" and "authornotemark" commands
% used to denote shared contribution to the research.
\author{Shayne Fletcher}
\affiliation{%
  \instituion{Digital Asset}
  \department{Language Engineering}
  %% \streetaddress{4 World Trade Center, 150 Greewhich Street, 47\textsuperscript{th} Floor}
  %% \city{New York}
  %% \state{NY}
  %% \postcode{10012}
  \country{USA}
}
\email{shayne.fletcher@daml.com}

% By default, the full list of authors will be used in the page headers. Often, this list is too long, and will overlap
% other information printed in the page headers. This command allows the author to define a more concise list
% of authors' names for this purpose.
\renewcommand{\shortauthors}{Shayne Fletcher}

%% \begin{figure}[H]
%% \centering
%% \includegraphics{img/DAML_330x89.png}
%% \end{figure}

\begin{abstract}
The language for automation of multiparty workflows.
\end{abstract}

%
% The code below is generated by the tool at http://dl.acm.org/ccs.cfm.
% Please copy and paste the code instead of the example below.
%

\begin{CCSXML}
<ccs2012>
<concept>
<concept_id>10003752.10003766.10003767.10003768</concept_id>
<concept_desc>Theory of computation~Algebraic language theory</concept_desc>
<concept_significance>500</concept_significance>
</concept>
<concept>
<concept_id>10003752.10003766.10003767.10003769</concept_id>
<concept_desc>Theory of computation~Rewrite systems</concept_desc>
<concept_significance>500</concept_significance>
</concept>
<concept>
<concept_id>10003752.10003753.10010622</concept_id>
<concept_desc>Theory of computation~Abstract machines</concept_desc>
<concept_significance>300</concept_significance>
</concept>
<concept>
<concept_id>10003752.10010070.10010111.10011735</concept_id>
<concept_desc>Theory of computation~Theory of database privacy and security</concept_desc>
<concept_significance>300</concept_significance>
</concept>
<concept>
<concept_id>10003752.10010124.10010125.10010127</concept_id>
<concept_desc>Theory of computation~Functional constructs</concept_desc>
<concept_significance>300</concept_significance>
</concept>
<concept>
<concept_id>10011007.10011006.10011041</concept_id>
<concept_desc>Software and its engineering~Compilers</concept_desc>
<concept_significance>500</concept_significance>
</concept>
<concept>
<concept_id>10011007.10011006.10011041.10011688</concept_id>
<concept_desc>Software and its engineering~Parsers</concept_desc>
<concept_significance>500</concept_significance>
</concept>
<concept>
<concept_id>10011007.10011006.10011041.10011048</concept_id>
<concept_desc>Software and its engineering~Runtime environments</concept_desc>
<concept_significance>300</concept_significance>
</concept>
<concept>
<concept_id>10002951.10002952</concept_id>
<concept_desc>Information systems~Data management systems</concept_desc>
<concept_significance>300</concept_significance>
</concept>
</ccs2012>
\end{CCSXML}

\ccsdesc[500]{Software and its engineering~Compilers}
\ccsdesc[500]{Software and its engineering~Parsers}
\ccsdesc[300]{Software and its engineering~Runtime environments}
\ccsdesc[100]{Information systems~Data management systems}
\ccsdesc[100]{Theory of computation~Algebraic language theory}
\ccsdesc[100]{Theory of computation~Rewrite systems}
\ccsdesc[100]{Theory of computation~Abstract machines}
\ccsdesc[100]{Theory of computation~Theory of database privacy and security}
\ccsdesc[300]{Theory of computation~Functional constructs}

\keywords{Haskell, GHC, compiler, business processes, open source}

%
% A "teaser" image appears between the author and affiliation information and the body
% of the document, and typically spans the page.
\begin{teaserfigure}
\includegraphics{img/DAML_330x89}
%% \includegraphics[width=\textwidth]{sampleteaser}
%%  \caption{Seattle Mariners at Spring Training, 2010.}
%%  \Description{Enjoying the baseball game from the third-base seats. Ichiro Suzuki preparing to bat.}
\label{fig:teaser}
\end{teaserfigure}

%
% This command processes the author and affiliation and title information and builds
% the first part of the formatted document.
\maketitle

\section{Introduction}
DAML is a Haskell derivative designed to model the flow of rights and obligations in multiparty business processes (of particular relevance to Distributed Ledger Technologies). Beyond introduction to DAML, attendees learn DAML compiler implementation techniques, in particular, how DAML uses GHC as a library. Additionally, attendees hear details of DAML's open source development model.

\section{Talk overview}
Digital Asset proudly develop DAML, a multiparty business processes automation language. In this talk we explore DAML's character as a dialect of Haskell augmented by concepts and syntax for contract modeling.

Attendees learn how the DAML compiler front end utilizes GHC as a library. This usage encompasses the Lexical Analysis, Renaming, Typechecking and Desugaring phases. Explanation of the \verb|ghc-lib| package (on Hackage since February 2019) is included.

Lastly, attendees hear about Digital Asset's open source development model which includes \verb|ghc-lib|, the Digital Asset GHC fork and DAML itself.

\end{document}
