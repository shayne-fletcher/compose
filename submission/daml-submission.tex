\documentclass[acmsmall]{acmart}
\settopmatter{printacmref=false} % Removes citation information below abstract
\renewcommand\footnotetextcopyrightpermission[1]{} % removes footnote with conference information in first column
\pagestyle{plain} % removes running headers
\usepackage[parfill]{parskip}

% defining the \BibTeX command - from Oren Patashnik's original BibTeX documentation.
\def\BibTeX{{\rm B\kern-.05em{\sc i\kern-.025em b}\kern-.08emT\kern-.1667em\lower.7ex\hbox{E}\kern-.125emX}}

% Rights management information.
% This information is sent to you when you complete the rights form.
% These commands have SAMPLE values in them; it is your responsibility as an author to replace
% the commands and values with those provided to you when you complete the rights form.
%
% These commands are for a PROCEEDINGS abstract or paper.
%\copyrightyear{2019}
%% \acmYear{2019}
\setcopyright{none}
%% \acmConference[Compose '19]{Woodstock '18: ACM Symposium on Neural Gaze Detection}{June 03--05, 2018}{Woodstock, NY}
%% \acmBooktitle{Woodstock '18: ACM Symposium on Neural Gaze Detection, June 03--05, 2018, Woodstock, NY}
%% \acmPrice{15.00}
%% \acmDOI{10.1145/1122445.1122456}
%% \acmISBN{978-1-4503-9999-9/18/06}

%
% These commands are for a JOURNAL article.
%\setcopyright{acmcopyright}
%\acmJournal{TOG}
%\acmYear{2018}\acmVolume{37}\acmNumber{4}\acmArticle{111}\acmMonth{8}
%\acmDOI{10.1145/1122445.1122456}

%
% Submission ID.
% Use this when submitting an article to a sponsored event. You'll receive a unique submission ID from the organizers
% of the event, and this ID should be used as the parameter to this command.
%\acmSubmissionID{123-A56-BU3}

%
% The majority of ACM publications use numbered citations and references. If you are preparing content for an event
% sponsored by ACM SIGGRAPH, you must use the "author year" style of citations and references. Uncommenting
% the next command will enable that style.
%\citestyle{acmauthoryear}

\begin{document}

\title{DAML : A Haskell based language for Digital Contracts}

%
% The "author" command and its associated commands are used to define the authors and their affiliations.
% Of note is the shared affiliation of the first two authors, and the "authornote" and "authornotemark" commands
% used to denote shared contribution to the research.
\author{Shayne Fletcher}
\affiliation{%
  \instituion{Digital Asset}
  \streetaddress{4 World Trade Center, 150 Greewhich Street, 47th Floor}
  \city{New York}
  \state{NY}
  \postcode{10012}
}

% By default, the full list of authors will be used in the page headers. Often, this list is too long, and will overlap
% other information printed in the page headers. This command allows the author to define a more concise list
% of authors' names for this purpose.
\renewcommand{\shortauthors}{Shayne Fletcher}

\begin{figure}[H]
\centering
\includegraphics{img/DAML_330x89.png}
\end{figure}

\begin{abstract}
Introducing DAML --- the language for fast automation of multiparty workflows.
\end{abstract}

%
% The code below is generated by the tool at http://dl.acm.org/ccs.cfm.
% Please copy and paste the code instead of the example below.
%
\begin{CCSXML}
<ccs2012>
<concept>
<concept_id>10011007.10011006.10011041</concept_id>
<concept_desc>Software and its engineering~Compilers</concept_desc>
<concept_significance>500</concept_significance>
</concept>
<concept>
<concept_id>10011007.10011006.10011041.10011048</concept_id>
<concept_desc>Software and its engineering~Runtime environments</concept_desc>
<concept_significance>300</concept_significance>
</concept>
<concept>
<concept_id>10002951.10002952</concept_id>
<concept_desc>Information systems~Data management systems</concept_desc>
<concept_significance>300</concept_significance>
</concept>
</ccs2012>
\end{CCSXML}

\ccsdesc[500]{Software and its engineering~Compilers}
\ccsdesc[500]{Software and its engineering~Runtime environments}
\ccsdesc[500]{Information systems~Data management systems}

\keywords{Haskell, GHC, business processes, compilers, open source}

%
% A "teaser" image appears between the author and affiliation information and the body
% of the document, and typically spans the page.
%%\begin{teaserfigure}
%%  \includegraphics[width=\textwidth]{sampleteaser}
%%  \caption{Seattle Mariners at Spring Training, 2010.}
%%  \Description{Enjoying the baseball game from the third-base seats. Ichiro Suzuki preparing to bat.}
%%  \label{fig:teaser}
%%\end{teaserfigure}

%
% This command processes the author and affiliation and title information and builds
% the first part of the formatted document.
\maketitle

\section{Introduction}
DAML is a Haskell derivative designed to model the flow of rights and obligations in multiparty business processes (particularly relevant in the context of Distributed Ledger Technologies). Beyond being introduced to DAML, attendees will learn about techniques used to implement the DAML compiler, in particular, how DAML uses GHC as a library. Additionally, attendees will hear about DAML's open source development model.

\section{Talk overview}
Digital Asset proudly develop DAML, a language used to define multiparty business processes for the purpose of automation. In this talk we explore DAMLs character as a dialect of Haskell extended with specific syntax and concepts for the contract modeling domain.

Attendees will be shared details on how the front end of the DAML compiler utilizes GHC as a library. This usage encompasses the Lexical Analysis, Renaming, Typechecking and Desugaring phases. This aspect of the talk will include an explanation of the recently open sourced \verb|ghc-lib| project (made first available on Hackage in February 2019).

Lastly, attendees will hear about Digital Asset's open source development model that includes all the source code covering the \verb|ghc-lib|, Digital Asset \verb|ghc| fork and DAML itself.

\end{document}
